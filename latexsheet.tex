\documentclass[8pt]{extarticle} % ...it lives! That's a useful ability...
% OK, so I shrank everything to 8pt font; make sure this is readable! 
% bear in mind that TMTOWTDI
 % \documentclass[8pt,landscape]{article}
\usepackage{multicol}
\usepackage{calc}
\usepackage{ifthen}
\usepackage[landscape]{geometry}
\usepackage{amsmath} % will this break anything? I sure hope not...
\usepackage{hyperref}
% I may want to:
%   * shrink the font further (done) 
%   * play around with the margins to get every last bit of space (done) 


% To make this come out properly in landscape mode, do one of the following
% 1.
%  pdflatex latexsheet.tex
%
% 2.
%  latex latexsheet.tex
%  dvips -P pdf  -t landscape latexsheet.dvi
%  ps2pdf latexsheet.ps


% If you're reading this, be prepared for confusion.  Making this was
% a learning experience for me, and it shows.  Much of the placement
% was hacked in; if you make it better, let me know...


% 2008-04
% Changed page margin code to use the geometry package. Also added code for
% conditional page margins, depending on paper size. Thanks to Uwe Ziegenhagen
% for the suggestions.

% 2006-08
% Made changes based on suggestions from Gene Cooperman. <gene at ccs.neu.edu>


% To Do:
% \listoffigures \listoftables
% \setcounter{secnumdepth}{0}
% ...this isn't Jeffrey's to-do list, FYI

% This sets page margins to .5 inch if using letter paper, and to 1cm
% if using A4 paper. (This probably isn't strictly necessary.)
% If using another size paper, use default 1cm margins.
\ifthenelse{\lengthtest { \paperwidth = 11in}} % I, Jeffrey, have modified this. 
	{ \geometry{top=.25in,left=.25in,right=.25in,bottom=.25in} }
	{\ifthenelse{ \lengthtest{ \paperwidth = 297mm}}
		{\geometry{top=1cm,left=1cm,right=1cm,bottom=1cm} }
		{\geometry{top=1cm,left=1cm,right=1cm,bottom=1cm} }
	}
% this does give me some more space...


% Turn off header and footer
\pagestyle{empty}
 

% Redefine section commands to use less space
\makeatletter
\renewcommand{\section}{\@startsection{section}{1}{0mm}%
                                {-1ex plus -.5ex minus -.2ex}%
                                {0.5ex plus .2ex}%x
                                {\normalfont\large\bfseries}}
\renewcommand{\subsection}{\@startsection{subsection}{2}{0mm}%
                                {-1explus -.5ex minus -.2ex}%
                                {0.5ex plus .2ex}%
                                {\normalfont\normalsize\bfseries}}
\renewcommand{\subsubsection}{\@startsection{subsubsection}{3}{0mm}%
                                {-1ex plus -.5ex minus -.2ex}%
                                {1ex plus .2ex}%
                                {\normalfont\small\bfseries}}
\makeatother

% Define BibTeX command
\def\BibTeX{{\rm B\kern-.05em{\sc i\kern-.025em b}\kern-.08em
    T\kern-.1667em\lower.7ex\hbox{E}\kern-.125emX}}

% Don't print section numbers
\setcounter{secnumdepth}{0}


\setlength{\parindent}{0pt}
\setlength{\parskip}{0pt plus 0.5ex}


% -----------------------------------------------------------------------

\begin{document}

\raggedright
\footnotesize
\begin{multicols}{5} % this needs to be played with, I have a lot of stuff to squeeze into a small space! 
% six columns seems to be a bit much, four gives everything some breathing room, 
% five seems to be a pretty good place to be right now. This will be re-evaluated once I add everything in. 

\newlength{\MyLen}


% multicol parameters
% These lengths are set only within the two main columns
%\setlength{\columnseprule}{0.25pt}
\setlength{\premulticols}{1pt}
\setlength{\postmulticols}{1pt}
\setlength{\multicolsep}{1pt}
\setlength{\columnsep}{2pt}

\begin{center}
     \Large{\textbf{Math 122L Cheat Sheet}} \\
\end{center}

\section{Pre-Calculus}
\subsection*{Assorted}
% \begin{multicols}{2} 
$ x^{m/n}=\sqrt[n]{x^m}=\left(\sqrt[n]{x}\right)^m$ \\ 
$\sqrt[n]{xy}=\sqrt[n]{x}\sqrt[n]{y}$ \\
Sum of the squares of the first $n$ positive integers: 
$\frac{n(n+1)(2n+1)}{6n^2}$ \\
For every real number $x$, 
$$\lim_{n \to \infty}\frac{x^n}{n!}=0$$ \\
% \end{multicols}
A function $f$ is \textbf{even }if $f(-x)=f(x)$ \\
A function $f$ is \textbf{odd }if $f(-x)=-f(x)$ \\
$$\lim_{x\to \infty}\tan^{-1}x=\frac{\pi}{2}$$ \\
$$\lim_{x\to -\infty}\tan^{-1}x=-\frac{\pi}{2}$$ \\

\subsection*{Trig Functions}
% \begin{multicols}{2}
$\csc\theta=\frac{1}{\sin\theta}$ \\
$ \sec\theta=\frac{1}{\cos\theta}$ \\
$1+\tan^2\theta=\sec^2\theta$ \\
$1+\cot^2\theta=\csc^2\theta$ \\
$\cos\theta=\cos(-\theta)$ \\
$\tan\left( -theta\right)=-\tan\left(\theta\right) $ \\
$\sin\left(\frac{\pi}{2}-\theta\right)=\cos\theta$ \\
$\cos\left(\frac{\pi}{2}-\theta\right)=\sin\theta$ \\
$\tan\left(\frac{\pi}{2}-\theta\right)=\cot\theta$ \\

% \end{multicols}
 $\sin(x+y)=\sin x\cos y+\cos x \sin y$\\ % this goes over the prescribed boundaries- I think I will put it outside the multicols environment. 
 $\sin(x-y)=\sin x \cos y-\cos x \sin y$ \\ 
 $\cos(x-y)=\cos x \cos y-\sin x \sin y$ \\ 
 $ \tan(x+y)=\frac{\tan x -\tan y}{1+\tan x \tan y}$ \\
 
 
\subsection{Trig Tables} 
$\begin{array}{cccc} 
 % x, sin, cos, tan
x & \sin & \cos & \tan \\ \hline
 0 & 0 & 1 & 0 \\ 
 \frac{pi}{6} & 0.5 & \frac{\sqrt{3}}{2} & \frac{1}{\sqrt{3}} \\
 \frac{\pi}{4} & \frac{1}{\sqrt{2}} & \frac{1}{\sqrt{2}} & 1 \\
 \frac{\pi}{3} & \frac{\sqrt{3}}{2} & 0.5 & \sqrt{3} \\
 \frac{\pi}{2} & 1 & 0 & (undefined) \\
 \end{array}
$
 
%  \section{Basic Differentiation Topics} 
 % Do this later on
\section{Derivatives and Other Fun Things} 
 \subsection{L'Hospital's Rule}
%  Suppose $f$ and $g$ are differentiable and $g'(x)\neq 0$ near $a$. Suppose $$\lim_{x\to a}f(x)=0\textrm{ and}\lim_{x\to a}g(x)=0 $$ 
%  or $$\lim_{x\to a}f(x)=
For some $f(x)$ and $g(x)$, if we have indeterminate forms of the type $\frac{0}{0}$, $\frac{\infty}{\infty}$, $0^0$, $\infty-\infty$, $\infty^0$, $1^\infty$  we can evaluate it as $$\lim_{x\to a} \frac{f(x)}{g(x)}=\lim_{x\to a}\frac{f'(x)}{g'(x)}$$ ...if the limit on the right side exists. 
\subsection{The Definite Integral/Riemann Sums}
The definite integral from $a$ to $b$ is defined as:
$$\int_a^bf(x)dx=\lim_{n\to \infty}\sum_{i=1}^n f(x\mbox{*})\Delta x$$
...where $x\mbox{*}$ is any point in the $i$th subinterval $[x_{i=1},x_i]$ and 
$\Delta x=\frac{b-a}{n}$ and $x_i=a+i\Delta x$
If the limit exists, the $f$ is \textit{integrable} on $[a,b]$. 
% I could add some more stuff from 5.2 if there's space/I care/I don't run out of time/I have nothing else to do, no scenarios being especially plausible. 
\subsection{Derivatives and Curves}
% this is from section 4.3: 
\textbf{The mean value theorem }states that, for some differentiable $f$ on the 
interval $[a,b]$, there is a number $c$ such that
$f'(c)=\frac{f(b)-f(a)}{b-a}$, or equivalently, 
$f(b)-f(a)=f('(c)(b-a)$ \\ 
\textbf{Concavity}for some $f(x)$ over $[a,b]$:
\begin{enumerate}
\item If $f''(x)>0$ for all $x$ in the interval, f is concave upward over interval. 
\item If $f''(x)<0$ for all $x$ in the interval, f is concave downward over interval. 
\end{enumerate}
The \textbf{second derivative test} states that for some $f''$ continous near $c$: 
\begin{enumerate}
\item $f'(c)=0 and (f''(x)>0, f$ has a local minimum at $c$. 
\item $f'(c)=0 and (f''(x)<0, f$ has a local maximum at $c$.  
\end{enumerate}
\textbf{Fundamental Theorem of Calculus (part 1)}: 
If $f$ if continuous on $[a,b]$ and $a\leq x\leq b$, then 
$g(x)=\int_a^xf(t)dt$ is an antiderivative of $f$. 
\textbf{Fundamental Theorem of Calculus (part 2)}: 
Supposing $f$ is continuous on $[a,b]$: 1
\begin{enumerate}
\item $g(x)=\int_a^xf(t)dt$, then $g'(x)=f(x)$ 
\item $\int_a^bf(x)dx=F(b)-F(a)$, where $F$ is any antiderivative of $f$. 
\end{enumerate}
% Should I bother to try and clean up the formatting for this? I guess I'll just have to see how much space I have...
% ...stopped this part at section 5.4

\subsection{Antiderivatives}

\begin{align*}
\int \sec ^2 u du &= \tan u +C \\
\int \csc ^2 u du &=-\cot u +C \\
\int \tan u du &=\ln \left| \sec u \right| +C \\
\int a^u du &=\frac{a^u}{\ln a}+C \\
\int \sin ^2 y du&=\frac{1}{2}u -\frac{1}{4}\sin 2u +C \\
\int \tan ^2 u du &=\tan u-u+C \\
% Inverse trig time! 
\int \sin^{-1}u du&=u \sin^{-1}u+\sqrt{1-u^2} \\
\int \cos^{-1}u du&=u \cos^{-1}u+\sqrt{1-u^2} \\
\int \tan ^{-1}u du&=u \tan ^{-1}u-\frac{1}{2}\ln (1+u^2)+C \\
\int u dv&=uv-\int v du \\ % I don't like how this is spaced improperly, but it's not a huge deal right now...
\int u \cos u du&=\cos u +u\sin u \\
\int u \sin u du &=\sin u -u\cos u \\
%
\end{align*}

\subsection{Derivatives} 
\begin{align*}
\frac{d}{dx}(\tan x)&=\sec^2 x \\
\frac{d}{dx}(\cot x)&=-\csc^2x \\
\frac{d}{dx} (\sec^{-1})&=\frac{1}{x\sqrt{x^2-1}} \\
\frac{d}{dx}(\tan ^{-1}x)&=\frac{1}{1+x^2} \\
\frac{d}{dx}(\cot ^{-1}x)&=-\frac{1}{1+x^2} \\
\frac{d}{dx}(\sin^{-1} x)&=\frac{1}{\sqrt{1-x^2}} \\
\frac{d}{dx}(\cos^{-1}x)&=-\frac{1}{\sqrt{1-x^2}} \\
\frac{d}{dx}(a^x)&=a^x\ln a \\
\frac{d}{dx}(\log_ax)&=\frac{1}{x\ln a} \\
\frac{d}{dx}\left[\frac{f(x)}{g(x)}\right]&=\frac{g(x)f'(x)-f(x)g'(x)}{\left[ g(x)\right]^2}
\end{align*}
\subsection{Average Value}
$$f_{ave}=\frac{1}{b-a}\int_a^bf(x)dx$$ 

\subsection{Substitution Rule}

If $g'$ is continuous on $[a,b]$ and $f$ is continuous on the range of $u=g(x)$, then
$$\int_a^bf(g(x))g'(x)dx=\int^g(b)_{g(a)}f(u)du$$
% this overflows its bounds. I gotta fix this somehow (shrink the font further?) 

\subsection{Approximate Integration}

\textbf{Midpoint rule:}
$$\sum_{i=1}^n f(\bar{x_n}\Delta x$$
where $\Delta x=\frac{b-a}{n}$ and 
$\bar{x_n}=\frac{1}{2}(x_{n-1}+x_n)$, that is, the midpoint of $[x_{n-1},x_n] $
The \textbf{trapezoidal rule:} add a left-hand sum and a right-hand sum and 
divide both sides by two. % this is not very mathematical...
\textbf{Error bounds:}
 Suppose $\left|f''(x)\right|\leq K$
for $a\leq x \leq b$. If $E_T$ and $E_M$
are the errors in the Trapezoidal and Midpoint rules, then 

$\left|E_T\right| leq\frac{K(b-a)^3}{12n^2}$  and % Does this need to be an equation? 
$\left| E_M\right| \leq \frac{K(b-a)^3}{24n^2}$

\textbf{Simpson's Rule:}
$ S_n=\frac{\Delta}{3}[f(x_0)+4f(x_1)+2f(x_2)+4f(x_3)+...
+2f(x_{n-2})+4f(x_{n-1})+f(x_n)]$ 
where $n$ is even and $\Delta x=\frac{b-a}{n}$ 

The \textbf{error bound for Simpson's rule} for $ \left| f^{(4)}(x)\right| \leq K$ for $a\leq x\leq b$ is given as 
$$\left| E_S \right|\leq \frac{K(b-a)^5}{180n^4}$$ 

\subsection{Improper Integrals}

% This is the definition of an improper integral of type 1 (as found on page (414)
If $\int_a^bf(x)dx$ exists for every number $t\leq a$, then 
$\int_a^\infty f(x)dx=\lim_{t\to \infty}\int_a^t f(x)dx$, provided the limit exists as a finite number. 

If  $\int_t^b f(x)dx$ exists for every number $t\leq q$, then 
$\int_{-\infty}^b f(x)dx=\lim_{t\to -\infty}\int_t^b f(x) dx$
...provided this limit exists as a finite number. The integrals $\int_a^\infty f(x) dx$ and $\infty_-\infty^b f(x) dx$ are called \textbf{convergent} if the corresponding limit exists and \textbf{divergent} if the limit does not exist. 

If both $\int_a^\infty f(x) dx$ and $\int_{-\infty}^af(x) dx$ are convergent, then we define
$\int_{-\infty}^{\infty}f(x)dx=\int_{-\infty}^a f(x)dx+\int_a^\infty f(x)dx$ % is this readable? 
Note that any real number can be used for $a$. 

The improper integral $\int_1^\infty \frac{1}{x^p} dx$ is convergent if $p>1$ and divergent if
$p\leq 1$. % is this necessary? Probably not...

% Note that there is more to add with regards to improper integrals (particularly on page 418), but 
% I think my time (and space) would be better spent working on the series stuff. 

\section{Probability}
The odds of a certain event happening (given $n$ chances) are given as \\
$(\mbox{winning Odds})(\mbox{losing odds})^{n-1}$

\subsection{PDFs}

(...PDF standing for probability density function...) 
Every continuous random variable $X$ has a \textbf{probability density function} $f$. 
This means that the probability that $X$ lies between $a$ and $b$ is found by integrating $f$ from $a$ to $b$: 
$$P(a\leq X \leq b)=\int_x^b f(x) dx$$

For some PDF $f$, 
$$\int_{-\infty}^{\infty}f(x) dx=1$$ % I think I can afford display mode 

The \textbf{mean} of any probability density function $f$ (which can be interpreted as the long-range average of the random variable $X$) is defined to be 
$$\mu=\int_{-\infty}^\infty xf(x)dx$$

 \textbf{Average value} is given as $\sum_{i=1}^n p_i  x_i$\\

\textbf{Expected value} is given as $\sum_{\mbox{all x}}x p(x)$, where $p(x)$ is the PDF 

\subsection{Normal Distributions}

Functions that can be modeled with a normal distribution have a PDF of the form
$$f(x)=\frac{1}{\sigma \sqrt{2\pi}}e^{-(x-\mu)^2/(2\sigma^2)}$$
...where $\sigma$ represents the \textbf{standard deviation}. 
For any normal distribution,
$$\int_{-\infty}^\infty \frac{1}{\sigma \sqrt{2\pi}}e^{-(x-\mu)^2/(2\sigma^2)}dx=1$$

% section on expected value? yes, but I will come back to this



\section{Center of Mass}
Density of a circle whose density changes with radius: 
$$\int_a^b2\pi xg(x)dx$$

Center of mass: 
$$\bar{x}=\frac{\mbox{Sum of Moments of Mass}}{\mbox{Total Mass}}$$
...so, for a triangle with point masses of $5gm, 3gm,$ and $1gm$ at $x=-10, x=1, x=2$ :
$$\bar{x}=\frac{5(-10)+3(1)+1(2)}{5+3+1}$$

For a region $R$ of constant density $\delta$, the center of mass of $R$ is given by the point $(\bar{x},\bar{y},\bar{z})$: 
$$\bar{x}=\frac{\int x \delta f(x)dx}{\textrm{Total Mass}}$$ \\
$$\bar{y}=\frac{\int y \delta f(y)dy}{\textrm{Total Mass}}$$ \\
$$\bar{z}=\frac{\int z \delta f(z)dz}{\textrm{Total Mass}}$$
% this section should be finished. 


\section{Series}
% add sequences in if there's space? We really haven't spent much time with sequenes...
\subsection{Convergence and Divergence}

If the sequence $s_n$ is convergent and $\lim_{n\to\infty}s_n=s$ exists as a real number, then the 
series $\sum a_n$ is \textbf{convergent}, that is, 
$$\sum_{n=1}^\infty a_n=s$$
If the sequence $s_n$ is \textbf{divergent}, then the series is divergent. 

If the series $\sum_{n=1}^\infty a_n$ is convergent, then $\lim_{n\to \infty}a_n=0$ 
\textbf{Note that the converse of this is not true in general}; if $\lim_{n\to \infty}a= 0$, we 
cannot conclude that $\sum a_n$ is convergent. 

\textbf{The test for divergence} states that if $\lim_{n\to \infty} a_n$ does not exist of if $\lim_{n\to \infty} a_n \neq 0$, then the series $\sum_{n=1}^\infty a_n$ is divergent. 

% leaving out (for the sake of space) theorem 8 on oage 571

% should I add something in about telescoping terms? 

Sum of a geometric series: $$\sum_{n=1}^\infty ar^n=\frac{a}{1-r}\mbox{ for } \left| r \right| <1$$
If $\left| r\right| \geq 1$, the series is divergent. 

The \textbf{integral test} states that the for some continuous, positive $f$ decreasing on $[1,\infty)$, 
and $a_n=f(n)$, then the series is convergent if and only if the improper integral 
$\int_1^\infty f(x)dx$ is convergent. 

The $p$-series, $\sum_{m=1}^\infty \frac{1}{n^p}$ is convergent if $p>1$ and divergent if $p\leq 1$

\textbf{Comparison test:} Suppose that $\sum a_n$ and $\sum b_n$ are series with positive terms. 
\begin{itemize} 
\item If $\sum b_n$ is convergent and $a_n\leq b_n$ for all $n$, then $\sum a_n$ is also convergent. 
\item If $\sum b_n$ is divergent and $a_n\geq b_n$ for all $n$, then $\sum a_n$ is also divergent. 
\end{itemize}

\textbf{The limit comparison test:} \\ % do I need to have a new line here? 
Suppose that $\sum a_n$ and $\sum b_n$ are series with positive terms; if 
$$\lim_{n\to \infty} \frac{a_n}{b_n}=c$$
...where $c$ is a finite number and $c>0$, then either both series converge or both series diverge. 

% skipping the remainder estimate for the integral test for now
\textbf{The alternating series test:} \\
If the series $\sum_{n=1}^\infty(-1)^{n-1}b_n$ (for $b_n>0$)
satisfies the conditions 
\begin{itemize}
\item $b_{n+1}\leq b_n$ for all $n$
\item $\lim_{n\to \infty} b_n=0$
\end{itemize}

...then the series is convergent. 

\textbf{Alternating series estimation theorem:} \\
If $s=\sum (-1)^{n-1}b_n$ is the sum of alternating series that satisfies the alternating series theorem, then
$$\left|R_n\right|=\left| s-s _n\right| \leq 4b_{n+1}$$

A series $\sum a_n$ is \textbf{absolutely convergent} if $\sum\left| a_n \right|$ is convergent. 
If a series is absolutely convergent, then it is convergent. 

The \textbf{ratio test} for a series $\sum_{n=1}^\infty a_n$ 
where $\lim_{n\to \infty}\left|\frac{a_{n+1}}{a_n}\right|=L$
\begin{enumerate}
\item If $L<1$, then the series is absolutely convergent
\item If $L>1$ or $L=\infty$, then the series is divergent
\item If $L=1$, the ratio test is inconclusive
\end{enumerate}

\subsection{Power Series} 
For a given power series $\sum^\infty_{n=0}c_n (x-a)^n$
there are only three possibilities: 
\begin{enumerate}
\item The series converges only when $x=a$ 
\item The series converges for all $x$ 
\item There is a positive number $R$ such that the series converges if $\left| x-a\right|<R$ 
and diverges if $\left| x-a \right|>R$. 
\end{enumerate}
...the number $R$ is called the \textbf{radius of convergence} of the power series. 





\section{Center of Mass}
Density of a circle whose density changes with radius: 
$$\int_a^b2\pi xg(x)dx$$

Center of mass: 
$$\bar{x}=\frac{\mbox{Sum of Moments of Mass}}{\mbox{Total Mass}}$$
...so, for a triangle with point masses of $5gm, 3gm,$ and $1gm$ at $x=-10, x=1, x=2$ :
$$\bar{x}=\frac{5(-10)+3(1)+1(2)}{5+3+1}$$

For a region $R$ of constant density $\delta$, the center of mass of $R$ is given by the point $(\bar{x},\bar{y},\bar{z})$: 
$$\bar{x}=\frac{\int x \delta f(x)dx}{\textrm{Total Mass}}$$ \\
$$\bar{y}=\frac{\int y \delta f(y)dy}{\textrm{Total Mass}}$$ \\
$$\bar{z}=\frac{\int z \delta f(z)dz}{\textrm{Total Mass}}$$
% this section should be finished. 

\subsection{Functions as Power Series}
The radius of convergence $R$ does not change when the power series is derived or integrated. 

% Section 8.7 un in my book
\subsection{Taylor's Theorem} states that if $f$ has a power series representation at $a$, that is, if 
$$f(x)=\sum_{n=0}^\infty c_n(x-a)^n$$ where $\left| x-a\right|<R$ then its coefficients are given by the formula $c_n=\frac{f^(n)(a)}{n!}$

A \textbf{Maclaurin series }is a Taylor series about $a=0$. 

If $f(x)=T_n(x)+R_n(x)$, where $T_n$ is the $n$th-degree Taylor polynomial of $f$ at $a$ and
$$\lim_{n\to \infty} R_n(x)=0$$
for $\left| x-a \right| <R$, then $f$ is equal to the sum of its Taylor series on the interval 
$\left| x-a \right|<R$. 

% \textbf{Taylor's Remainder Theorem}: 
\subsection{Taylor's Inequality:}
 if $\left| f^{n+1}(x)\right| \leq M$, 
% for $\left| x-a \right| \leq d$, 
 then the remainder $R_n(x)$ of the Taylor series satisfies the inequality 
$\left| R_n(x) \right|\leq \frac{M}{(n+1)!}\left| x-a\right| ^{n+1}$ 
for $\left| x-a \right| \leq d$. 
Find the largest possible value of $M$ within the range of $x$ values. Find the largest possible value of $x$ 
when computing $\left| x-a \right|^{n+1}$

% I might be able to shrink this down if I'm short on space

% should I know about the binomial series? I need to ask Mike about this tonight! 
\subsection{Important Maclaurin Series:}
% setting this up as an array, could also do as a set of equations

% Ought I to give the first few terms of each series? that might prove useful on the test...
% ...yes, if there's room. 
\begin{align*} 
\frac{1}{1-x}&=\sum_{n=0}^\infty x^n \\
e^x &=\sum_{n=0}^\infty \frac{x^n}{n!} \\ % can I add comments here? yes
\sin x &=\sum_{n=0}^\infty (-1)^n \frac{x^{2n+1}}{(xn+1)!} \\
\cos x &=\sum_{n=0}^\infty (-1)^n \frac{x^{2n}}{(2n)!} \\
\tan ^{-1} x&=\sum_{n=0}^\infty (-1)^n \frac{x^{2n+1}}{2n+1} \\
\ln (1+x)&=\sum_{n=1}\infty (-1)^{n-1} \frac{x^n}{n} \\
% (1+x)^k=\sum^\infty_{n=0} % I don't really know how to use the binomial theorem thin
% and I think I need to figure that out pretty soon...
\end{align*}

% Section 8.8
Roots can be approximated using Taylor polynomials; for the cube root of $x$, 
$f(x)=x^{1/3}$; we can treat this just as we would treat any other Taylor polynomial (by finding the coefficients with derivations).

 

\section{Fourier Series}
Functions with 
periods $-\pi\leq x\leq \pi$.:
% \textbf{Fourier series} have the form 
% $$f(x)=
$$a_0+\sum_{n=1}^\infty\left( a_n \cos nx+b_n \sin nx\right)$$ % fix formatting here? 
% The name of the game is to find a function for the coeffecients. 

$$a_0=\frac{1}{2\pi}\int_{-\pi}^\pi f(x)dx$$
$$a_n=\frac{1}{\pi}\int_{-\pi}^\pi f(x)\cos nx dx$$
$$b_n=\frac{1}{\pi}\int_{-\pi}^\pi f(x)\sin nx dx$$

If $f$ is a piecewise continuous on $[-L,L]$, the Fourier series is defined as: 
$$a_0+\sum_{n=1}^\infty\left[
a_n\/cos \left(\frac{n\pi x}{L}\right)+
b_n \sin\left(\frac{n \pi x}{L}\right)\right]
$$
% formatting is all weird, I'm going to have to deal with this a little later...
% ...I might trying using fewer columns, we'll see how things are as far as space
Coefficients are defined for $n\geq 1$ as
$$a_0=\frac{1}{2L}\int_{-L}^L f(x) dx$$
$$a_n=\frac{1}{L}\int_{-L}^L f(x) \cos \left( \frac{n\pi x}{L}\right)dx$$
$$b_n=\frac{1}{L}\int_{-L}^L f(x) \sin \left( \frac{n\pi x}{L}\right)dx$$

% If $f$ is a periodic function with period $2\pi$ and $f$ and $f'$ are piecewise continuous on $[-\pi,\pi]$, then the Fourier series is convergent.

The \textbf{sum of the Fourier series }is equal to $f(x)$  at all numbers $x$ where $f$ is continuous. At the numbers $x$ where $f$ is discontinuous, the sum of the Fourier series is defined as
$\frac{1}{2}\left[(x^+)+f(x^-)\right]$




\section*{Differential Equations and Friends }
\subsection*{Basics}
% I skipped 7.2 of the book. 

Differential Equations dictate the rate of change of a function. $\frac{dy}{dt}=k$ means the the function $y$ changes at a constant rate $k$. 

A \textbf{separable equation }is a first-order differential equation in which the expressions can be factored as a function of $x$ times a function of $y$, that is, $\frac{dy}{dx}=g(x)f(y)$
% \subsection*{Applications of Differential Equations}
% Chapter 7.4 in book 
\textbf{Euler's Method:}
Approximate values for the solution of the initial-value problem $y=F(x,y),y(x_0)=y_0$, with step size $h$, at $x_n=x_{n-1}$ are given as 
$$y_n=y_{n-1}+hF(x_{n-1},y_{n-1}), n=1,2,3...$$

\iffalse 
\begin{enumerate}
\item Assume a function $\frac{dy}{dx}=f(x)$ with some given $y_0$ and a step size of $h$
\item Add $h\times f(y_0)$ to $y_0$
\item Repeat as needed; $y$ 
\end{enumerate}
\fi 
The solution of the initial-value problem $\frac{dy}{dt}=ky$ where $y(0)=y_0$ is given  as $y(t)=y_0e^{kt}$

Newton's Law of Cooling States that $\frac{PdT}{dt}=k(T-T_s)$, where $k$ is a constant, $T\mbox{ is the temperature and } T_0 \mbox{ is the original temperature.}$

\subsection*{Population Growth Models}
% This is chapter 7.5 in the book. 
The logistic differential equation (one of the simplest models for logistic growth) is given as $$\frac{dP}{dt}=kP\left(1-\frac{P}{M}\right)$$ where $P$ is the population and $M$ is the carrying capacity. 

The solution to the logistic equation is given as $P(t)=\frac{M}{1+Ae^{-kt}}\mbox{where } A=\frac{M-P_0}{P_0}$ 

% 
% Template originally created in 2014 by Winston Chang

% \href{http://www.stdout.org/~winston/latex/}{http://www.stdout.org/$\sim$winston/latex/}


\end{multicols}
\end{document}
